% Linux emoji configuration (LuaLaTeX)
\usepackage{fontspec}
\usepackage{adjustbox}
\usepackage{xcolor}
\usepackage{colortbl}

% Define pandocbounded command if not available
\providecommand{\pandocbounded}[1]{#1}

\renewcommand{\familydefault}{\sfdefault}


% Set main fonts with full Unicode support including CJK
\setmainfont{Noto Sans}[
  BoldFont=Noto Sans Bold,
  ItalicFont=Noto Sans Italic,
  BoldItalicFont=Noto Sans Bold Italic
]

% Set monospace font
\setmonofont{DejaVu Sans Mono}[Scale=0.9]

% Set emoji font to Noto Color Emoji with HarfBuzz renderer for color support
\newfontfamily\emojifont{Noto Color Emoji}[Renderer=HarfBuzz]

% Set Japanese font for CJK characters
\newfontfamily\japanesefont{Noto Sans CJK JP}[Scale=1.0]

% Command to render emoji (generic wrapper)
\newcommand{\emoji}[1]{{\emojifont #1}}

% Command to render Japanese text
\newcommand{\jp}[1]{{\japanesefont #1}}

% Map Unicode characters to emoji font
\usepackage{newunicodechar}

% Standard Emojis - Using Noto Color Emoji
% LuaLaTeX with HarfBuzz usually handles these automatically if the font is set as main font or fallback,
% but explicit mapping ensures they use the correct font family.

% Explicit mappings for emojis used in documentation
\newunicodechar{📚}{\emoji{📚}}
\newunicodechar{🧠}{\emoji{🧠}}
\newunicodechar{🏗}{\emoji{🏗}}
\newunicodechar{⚙}{\emoji{⚙}}
\newunicodechar{🦀}{\emoji{🦀}}
\newunicodechar{🤖}{\emoji{🤖}}
\newunicodechar{✅}{\emoji{✅}}
\newunicodechar{❌}{\emoji{❌}}
\newunicodechar{⚠}{\emoji{⚠}}
\newunicodechar{🔧}{\emoji{🔧}}
\newunicodechar{💻}{\emoji{💻}}
\newunicodechar{🔒}{\emoji{🔒}}
\newunicodechar{⚡}{\emoji{⚡}}
\newunicodechar{🧪}{\emoji{🧪}}
\newunicodechar{📊}{\emoji{📊}}
\newunicodechar{🚀}{\emoji{🚀}}
\newunicodechar{🎉}{\emoji{🎉}}
\newunicodechar{🔄}{\emoji{🔄}}
\newunicodechar{🎯}{\emoji{🎯}}
\newunicodechar{💡}{\emoji{💡}}
\newunicodechar{🔥}{\emoji{🔥}}
\newunicodechar{⭐}{\emoji{⭐}}
\newunicodechar{🌟}{\emoji{🌟}}
\newunicodechar{💎}{\emoji{💎}}
\newunicodechar{🛡}{\emoji{🛡}}
\newunicodechar{🔑}{\emoji{🔑}}
\newunicodechar{🎨}{\emoji{🎨}}
\newunicodechar{💋}{\emoji{💋}}
\newunicodechar{🔷}{\emoji{🔷}}
\newunicodechar{🔶}{\emoji{🔶}}
\newunicodechar{🔴}{\emoji{🔴}}
\newunicodechar{🟢}{\emoji{🟢}}
\newunicodechar{🏆}{\emoji{🏆}}
\newunicodechar{🥇}{\emoji{🥇}}
\newunicodechar{🥈}{\emoji{🥈}}
\newunicodechar{🥉}{\emoji{🥉}}

% Mathematical symbols - use main font or specific math font if needed,
% ensuring they don't break if present in emoji font.
\newunicodechar{≈}{≈}
\newunicodechar{→}{→}
\newunicodechar{≤}{≤}
\newunicodechar{√}{√}
\newunicodechar{↔}{↔}
\newunicodechar{⋆}{⋆}
\newunicodechar{★}{\emoji{★}}
\newunicodechar{◇}{◇}
\newunicodechar{░}{░}

% Japanese characters - use Japanese font
\newunicodechar{ス}{\jp{ス}}
\newunicodechar{す}{\jp{す}}
\newunicodechar{カ}{\jp{カ}}
\newunicodechar{つ}{\jp{つ}}
\newunicodechar{え}{\jp{え}}
\newunicodechar{レ}{\jp{レ}}
\newunicodechar{サ}{\jp{サ}}
\newunicodechar{け}{\jp{け}}
\newunicodechar{へ}{\jp{へ}}
\newunicodechar{ワ}{\jp{ワ}}
\newunicodechar{さ}{\jp{さ}}
\newunicodechar{ほ}{\jp{ほ}}
\newunicodechar{き}{\jp{き}}
\newunicodechar{て}{\jp{て}}
\newunicodechar{ソ}{\jp{ソ}}
\newunicodechar{く}{\jp{く}}
\newunicodechar{い}{\jp{い}}
\newunicodechar{と}{\jp{と}}
\newunicodechar{の}{\jp{の}}
\newunicodechar{メ}{\jp{メ}}
\newunicodechar{む}{\jp{む}}
\newunicodechar{ル}{\jp{ル}}
\newunicodechar{ト}{\jp{ト}}
\newunicodechar{わ}{\jp{わ}}
\newunicodechar{ツ}{\jp{ツ}}
\newunicodechar{セ}{\jp{セ}}

% Other symbols
\newunicodechar{♫}{\emoji{♫}}
\newunicodechar{☎}{\emoji{☎}}

% Greek letters
\newunicodechar{α}{α}
\newunicodechar{δ}{δ}
\newunicodechar{φ}{φ}
\newunicodechar{τ}{τ}
\newunicodechar{λ}{λ}
\newunicodechar{π}{π}
\newunicodechar{Δ}{Δ}
\newunicodechar{Ε}{Ε}
\newunicodechar{ε}{ε}

% Box-drawing characters
\newunicodechar{├}{+}
\newunicodechar{─}{-}
\newunicodechar{│}{|}
\newunicodechar{└}{+}
\newunicodechar{┌}{+}
\newunicodechar{┐}{+}
\newunicodechar{┘}{+}
\newunicodechar{┬}{+}
\newunicodechar{┴}{+}
\newunicodechar{┼}{+}
\newunicodechar{━}{=}
\newunicodechar{┃}{|}
\newunicodechar{┏}{+}
\newunicodechar{┓}{+}
\newunicodechar{┗}{+}
\newunicodechar{┛}{+}

% Handle variation selectors
\newunicodechar{️}{}  % Variation Selector-16